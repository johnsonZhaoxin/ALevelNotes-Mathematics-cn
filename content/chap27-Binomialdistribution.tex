\chapter{概率分布模型}
数学不光光是完成计算,更重要的在于对同一类问题的概括,而这些概括得出来的结果就是各种各样的定理Theorem,小定理Lemma,推导出来的二级引理Collary,还有各种各样的数学模型。

在这个章节当中,我们主要探究离散随机变量的三种常见分布模型,分别是\gls{0-1distr},\gls{bidistr}和\gls{geodistr}

\section*{学习目标}
\begin{todolist}
	\item 理解0-1分布模型,二项分布模型,几何分布模型
	\item 判断每种场景下适合的模型
	\item 求算二项分布的概率,期望和方差
	\item 求算几何分布的概率,期望和方差
\end{todolist}
\clearpage

\section{0-1分布}
\subsection*{定义}
如\ref{subsec:Expectation}当中提到,0-1分布就是一个很简单的分布模型,也被称之为\gls{bernoullidistr}。其随机变量$X$只有两个取值,0或者1。我们可以称$X=1$为成功,概率为$p$。 失败的概率为$q=1-p$。
比如一个灯泡能够连续照明超过5000小时计作成功,概率为$0.3$\\
又或者是抛掷一个硬币,正面朝上计作成功,概率为$0.5$\\
再比如,下雨天的时候赶地铁,到地铁站就有车计作成功,概率为$0.01$
\subsection*{期望和方差}
利用概率分布表,任意0-1分布的期望求算公式为:
\[
	E(X)=0\cdot (1-p) +1 \cdot p = p
\]
因此,方差的求算公式为:
\[
	Var(X) = 0^2\cdot (1-p) +1^2 \cdot p - p^2=p(1-p)
\]
虽然考试当中从来不考察0-1分布,但是它是后续两种分布的重要基石。需要做到快速分辨0-1分布,以及确定对应的概率。
\clearpage

\section{二项分布}
这种分布模型是考试的重点,是将伯努利实验重复做$n$次。用随机变量$X$表示在这$n$次\gls{trial}中成功的次数。

\subsection*{定义}
沿用上面的例子,\\
那么20个灯泡之中,有多少个灯泡可以持续照明超过5000小时。\\
抛100个硬币,或者抛100次硬币,多少次正面朝上\\
在157天下雨天当中,到站台就有车的天数等等。

这些都是可以看成是一个二项分布的。
二项分布的\emph{特征/满足的要求}是:
\begin{itemize}
	\item 试验重复的次数$n$是固定值
	\item 每一次试验的结果都只有两个,0-1
	\item 针对于每一次的成功的概率都是$p$,失败的概率为$1-p$或者$q$
	\item 每一次试验之间的结果都是互相\emph{独立}的
\end{itemize}

\subsection*{概率求算}
当已知一个随机变量服从二项分布的时候,会采用如下的手段进行标记:
\[
	X \sim B(n,p)
\]
其中含义也非常明确。字母$B$代表Binomial,括号里的两个参数代表重复实验的次数$n$,以及在一次实验中成功的概率$p$。如此精炼符号系统其实也造成了一定程度的理解。反正最简洁的方法就是将随机实验当成抛硬币$n$次,正面朝上的概率为$p$,表示总共有$X$次正面朝上。

那么接下来就是对这个随机变量的进行概率求算了,取$n=5$作为一个例子。不难发现,$X=0$和$X=5$的求算是最简单的,分别是$(1-p)^5$和$p^5$。那么我们再求算一下$X=1$的概率,其含义是只有一次成功的概率。因此必定有四次失败。这五次的概率因为都是独立事件,所以概率可以使用乘法法则。$p\cdot (1-p)^4$。慢着,好像哪里不对,求5次当中成功了1次,但是这一次未必是第一次啊。因此还可以有$(1-p)\cdot p \cdot (1-p)^3 + (1-p)^2\cdot p \cdot (1-p)^2+(1-p)^3\cdot p \cdot (1-p)+(1-p)^4\cdot p$。分别表示第2、3、4、5次成功的概率。而将这些加起来得到$5\cdot p \cdot (1-p)^4$。其中的这个5不就是和二项展开式当中的$\binom{5}{1}$是完全一样的意义吗!因此这就是这种分布叫做\emph{二项}分布的原因了。类似的$X=4$的概率就是$\binom{5}{4}\cdot p^4\cdot(1-p)^1$。

所以,以$n=5$为例,我们就可以列出$X\sim B(5,p)$的概率分布表,如下:
\begin{table}[H]
\centering  
\begin{tblr}{
	width=0.9\textwidth, %调整表格宽度
	colspec={|l|[2pt]c|c|c|c|c|c|},
	hspan=even, %设置一样长度
	vspan=even,
	}
\hline 
$x$ & $0$ & $1$ & $2$ & $3$ & $4$ & $5$ \\
\hline
$P(X=x)$ & $\binom{5}{0} p^0\cdot (1-p)^5$ & $\binom{5}{1} p^1\cdot (1-p)^4$ & $\binom{5}{2} p^2\cdot (1-p)^3$\\ \hline
$x$ & $3$ & $4$ & $5$\\ \hline
$P(X=x)$ & $\binom{5}{3} p^3\cdot (1-p)^2$ &$\binom{5}{4} p^4\cdot (1-p)^1$ & $\binom{5}{5} p^5\cdot (1-p)^0$ \hline
\end{tblr}
\end{table}

这个表达式都可以用类似于之前的二项展开式写出来
\[
	P(X=r) = \binom{n}{r}\cdot p^r\cdot (1-p)^{n-r} \qquad 0\leqslant r \leqslant n
\]

这就是二项分布的定义求算,所以,如果一旦确定了某种实验属于二项分布,并且确定了两个参数——次数和概率。就可以直接套用公式进行概率的求算了。

\begin{ExampleBox}
Screws are sold in packets of 15. Faulty screws occur randomly. A large number of packets are tested for faulty screws and the mean number of faulty screws per packet is found to be 1.2. Find the probability that a packet cotians at most 2 faulty screws.\\
\makebox{}\hfill Adapted from 2014 winter qp61 Q5
\tcblower
首先根据平均每包当中坏的螺丝数量为1.2个,可以得知,任何一个螺丝是坏的概率为$\frac{1.5}{12}=0.125$。这个哪怕基于常识也是可以确定的,不过后面在\ref{subsec:EandVar}当中会继续讨论。因此就可以认为$X$代表15个螺丝当中坏的螺丝数目。因此$X\sim B(15,0.125)$

由于此题问最多两个坏螺丝,因此$P(X\leqslant2 )= P(X=0)+P(X=1)+P(X=2)$将0个,1个,2个坏螺丝出现的概率相加。因此最后结果为
\[
	0.875^15+15\times 0.125\times 0.875^{14}+105\times 0.125^2\times 0.875^{13} = 0.887
\]
\end{ExampleBox}

\subsection*{二项分布的期望与方差}
\label{subsec:EandVar}
任意随机变量都可以求算期望和方差,二项分布也不例外

\subsubsection*{期望求算}
因此沿用随机变量的期望求算公式,可以得到化简之后的二项分布的期望公式:
\[
	E(X)/\mu = np
\]
因此,在刚才的那条题目当中,实际上1.2就是15个螺丝当中坏的数目的平均值也就是随机变量的期望,所以我们才反用该关系,求算1个螺丝是坏的概率$p = \frac{E(X)}{n}=0.125$。考试当中有的时候也是这样,并不直接给出概率,而是通过给出期望来反求概率,要能够做到识别并能够运用该公式

\subsubsection*{方差}
沿用随机变量的方差求算公式,可以得到化简之后的二项分布的方差公式:
\[
	Var(X)/\sigma^2 =np(1-p) 
\]

\begin{ExampleBox}
考试当中并不要求理解两个公式,但是笔者还是认为能够自己尝试推导一下会更好。因此在这里给出证明过程。
\tcblower
\begin{flalign*}
	E(X) &=0\cdot ^nC_0\cdot (1-p)^n +1\cdot ^nC_1\cdot p \cdot (1-p)^{n-1} +\ldots + n\cdot ^nC_n\cdot p^n \\
		 &=p\cdot [1\cdot^nC_1\cdot (1-p)^{n-1} +2\cdot^nC_2\cdot p\cdot (1-p)^{n-2}+\ldots+n\cdot ^nC_n\cdot p^{n-1}]\\
		 &=np\cdot [^{n-1}C_0\cdot(1-p)^n+^{n-1}C_1\cdot p \cdot(1-p)^n+\ldots +^{n-1}C_{n-1}\cdot p^{n-1}]\\
		 &=np\cdot [p+(1-p)]^{n-1}\\
		 &=np
\end{flalign*}
这里,我们利用了二项展开系数的一个小性质:
\[
	k\cdot ^n C_k = n\cdot ^{n-1} C_{k-1}
\]
难度不大,可以利用定义或者数学归纳法进行证明。

因此可以证明二项分布的期望为$E(X)=np$

下一步,证明方差为$Var(X)=np\cdot(1-p)$
\begin{flalign*}
	E(X) &=0^2\cdot ^nC_0\cdot (1-p)^n +1^2\cdot ^nC_1\cdot p \cdot (1-p)^{n-1} +\ldots + n^2\cdot ^nC_n\cdot p^n -(np)^2\\
	&=\left(\sum_{k=1}^{n} k^2\cdot^nC_k\cdot p^k \cdot (1-p)^{n-k} \right)- (np)^2\\
	&=\left(\sum_{k=1}^{n} k\cdot^nC_k\cdot p^k \cdot (1-p)^{n-k} +\sum_{k=1}^{n} (k^2-k)\cdot^nC_k\cdot p^k \cdot (1-p)^{n-k}\right)-(np)^2\\
	&=np +\left(\sum_{k=1}^{n} (k-1)k\cdot^nC_k\cdot p^k \cdot (1-p)^{n-k}\right)-(np)^2\\
	&=np + np\left(\sum_{k=1}^{n} (k-1)\cdot^{n-1}C_{k-1}\cdot p^{k-1} \cdot (1-p)^{n-k}\right)-(np)^2)\\
	&=np + np(n-1)p\left(\sum_{k=1}^{n}\cdot^{n-2}C_{k-2}\cdot p^{k-2} \cdot (1-p)^{n-k}\right)-(np)^2\\
	&=np +n(n-1)p^2\cdot [p+(1-p)]^{n-2} -(np)^2\\
	&=np +n^2p^2-np^2-n^2p^2\\
	&=np(1-p)
\end{flalign*}
QED

方差的求算较为复杂,连续利用了两次$k\cdot ^nC_k = n \cdot ^{n-1}C_{k-1}$,为了方便理解,取较小的$n$值,进行推导。
\end{ExampleBox}
\clearpage


\section{几何分布}
几何分布和二项分布非常类似,都是基于重复执行0-1实验产生的随机变量。
\subsection*{定义}
与二项分布不同的地方在于,第一个是几何分布并不是执行固定次数的试验,而是不停地执行该试验直至成功为止。第二个是几何分布的随机变量不代表成功的次数,而是代表直至第一次成功总共所需的次数。第三个也就是带来的不同结果,二项分布的最小值可以为0,但是几何分布的最小值是1。二项分布存在有限个结果,最大为$n$,但是几何分布是无限的结果,因为可以重复无数多次依然不能成功。

继续沿用上面的例子
不停的测灯泡,直至得到一个可以持续照明超过5000小时的灯泡才停止实验。\\
连续地抛硬币,直至正面朝上停止,需要抛掷的次数为$X$\\
下雨天赶地铁,直至某一天到达站台上就可以直接上车,记录下这是第多少天。
这些情况都是属于几何分布。

因此可以计作
\[
	X\sim Geo(p)
\]
观察这种标记手段,Geo是Geometrical Distribution的缩写,$p$依旧是基础的0-1分布当中,成功一次的概率。但是仅有这一个参数,再回顾一遍,因为几何分布当中的随机变量代表第一次成功所用的次数。

\subsection*{概率求算}
按照既定流程,我们希望得到几何分布的概率大小。开始推导
最少要1次才能成功,因此概率表达为$P(X=1)=p$
第二次才能成功,$P(X=2)=(1-p)p$。好了,在这里就产生一个需要注意的点了,$p(1-p)$能不能算入第二次才成功?答案很明显是不能的,因为在这里并不是说成功一次失败一次。而是第一次试验必须要失败,然后才能执行第二次试验,并且成功。因此只能是$(1-p)p$。

讨论$X=3$,$X=4$等情况下的概率分别为$(1-p)^2\cdot p$和$(1-p)^3\cdot p$,不难发现,其实就是前面的$n-1$次都是失败的,然后最后一次是成功的。利用\textbf{乘法法则}就可以得到几何分布的概率表达式
\[
	P(x=k)=(1-p)^{k-1}\cdot p \qquad k\geqslant 1
\]
而这个表达式已出现,聪明的小伙伴就明白为什么这种分布叫做几何分布了。我们把这个概率分布表写下来。
\begin{table}[H]
\centering
\begin{tblr}{
	colspec={|l|[2pt]c|c|c|c|c|c|},
	hspan=even, %设置一样长度
	vspan=even,
	}
\hline 
$x$ & 1 & 2 & 3 & 4 & 5 & $\ldots$  \\
\hline
$P(X=x)$ & $p$ & $(1-p)p$ & $(1-p)^2p$ & $(1-p)^3p$ & $(1-p)^4p$ & $\ldots$  \\
\hline
\end{tblr}
\end{table}

随着随机变量的取值逐一增大,对应的概率\textbf{恰好}构成了一个以$p$为首项,以$1-p$作为公比的\textbf{Geometric Progression}!所以,这个恰好并不是巧合。就是因为每一次试验之间互相独立,所以可以使用乘法法则。并且由于指定只能是最后一次成功,因此没有其他的可能性。只能将$p$放在最后一次的位置上。借此就形成了几何分布。

\subsection*{几何分布的期望与方差}
还是常规流程,求算这种分布的期望与方差
\subsubsection*{期望求算}
可能有的人想,套公式就行了,so easy!这么说好像也对,但是我们再看一遍几何分布的概率分布表,利用$\sum x_i\cdot p_i$几何分布的随机变量$X$并不是有限的,是无限的!

因此我们实际上计算的是一个无限求和,写成如下的表达式
\[
	\sum_{k=1}^{\infty} k\codt p(1-p)^{k-1}=1\cdot p +2\cdot p(1-p)+3\cdot p(1-p)^2+\ldots 
\]
这种表达式叫做级数Series,考纲当中并不要求掌握求算,因此我们只需要知道这个表达式的计算结果是$\frac{1}{p}$即可。因此,几何分布的期望求算公式为:
\[
	E(X)/\mu = \frac{1}{p}
\]


\subsubsection*{无限多的几何分布概率}
虽然上面的无限多项求和不需要掌握,但是我们需要验证一下之前随机变量表的一个性质:所有概率之和为$1$。从这个角度稍微理解一下无限多项的级数求和的特征
\begin{flalign*}
	\sum_{k=1}^{\infty} p(1-p)^{k-1} &= p+p(1-p)+p(1-p)^2+\ldots\\
	&=\frac{p\cdot (1-r^n)}{1-r}\\
	&=\frac{p}{1-(1-p)}\cdot \left( 1- (1-p)^\infty\right)\\
\end{flalign*}
这不就是之前的等比数列的sum to infinity吗!因此为$1$。我们借用了这个来理解一下无限多项的求和的概念。

\subsubsection*{几何分布的方差}
还是同期望一样,几何分布的方差求算的难点是利用了无限多项的级数求和。这个部分并不需要掌握。考试当中也仅仅考到几何分布的期望就可以了。在这里给出公式,如果感兴趣可以查看\href{https://www.cs.cornell.edu/courses/cs280/2008sp/280wk11_x4.pdf}{几何分布的期望的推导},然后用同样的方式可以推导方差。
\[
	Var(X)/\sigma^2 = \frac{1-p}{p^2}
\]
考试当中并不考察几何分布的方差,因此这一小节仅作为阅读材料。

\clearpage

\section{例题应用}
S1的考试当中,二项分布和几何分布是重点内容。需要能够做到以下的几点
\begin{todolist}
	\item 求算某一种二项分布取特定值的概率
	\item 求算二项分布当中至少,至多等类型的概率
	\item 求算二项分布的期望或者方差,或者已知期望方差概率反求试验次数$n$和概率$p$
	\item 求算几何分布取特定值的概率
	\item 求算几何分布当中,至少,至多等类型的概率
	\item 求算几何分布的期望,或者已知期望反求概率
	\item 将几何分布和其他的概率知识点结合的考察,例如求算$P(X=3|X\leqslant4)$等条件概率
\end{todolist}

\subsection*{二项分布例题}
\begin{ExampleBox}
In a certain country, 68\% of households have a printer. Find the probability that, in a random sample of 8 households, 5, 6 or 7 households have a printer.\\
\makebox{}\hfill Adapted from 2015 summer qp61 Q6
\tcblower
分析题目,不难发现是一个经典的二项分布模型。其中基础的0-1分布为无打印机和有打印机,概率分别为$0.32$和$0.68$。而这个模型当中,重复的次数为$8$次。因此可以记$X$为这八家当中有打印机的数量。显然$X\sim B(8,0.68)$\\

因此题目当中所求的5,6,7就直接用二项分布的概率求算公式
\[
	^8C_5 \times 0.68^5 \times 0.32^3+^8C_6 \times 0.68^6 \times 0.32^2+^8C_7 \times 0.68^7 \times 0.32^1 = 0.7224
\]
\end{ExampleBox}
\clearpage

\begin{ExampleBox}
In a certain country, on average one student in five has blue eyes.For a random selection of $n$ students, the probability that none of the students has blue eyes is less than $0.001$. Find the least possible value of $n$.
\makebox{}\hfill Adapted from 2013 summer qp63 Q4
\tcblower
此题是经典的已知概率,求算参数的题目。依旧是一个二项分布。记$X$为这$n$个学生当中有蓝眼睛的人数。那么$X\sim B(n,\frac{1}{5})$。其中None说明$P(X=0)<0.001$。

因此列出概率求算公式$^nC_0 \cdot(\frac{4}{5})^n < 0.001$。利用指对数运算,可以求出
\[
	n>\log_{0.8} 0.0001 = 30.95
\]
因此$n$的最小值为31人
\end{ExampleBox}
无论形式怎么多变,只要能够确定模型,明确相关的条件信息,带入到对应的公式当中,就必然能够做对这些题目。



\subsection*{几何分布例题}
\begin{ExampleBox}
In order to start a board game, each player rolls fair six-sided dice, one at a time, until a $6$ is obtained. Let $X$ be the number of goes a player takes to start the game. Explain what does $X>4$ means in this scenario and find $P(X>4)$
\tcblower
此题的场景不就是一个飞行棋吗?扔出6才可以出动飞机。因此$X\sim Geo(\frac{1}{6})$,这就是将题意翻译成数学语言的结果。
那么$X>4$可以有两种解释,第一种,扔5次,6次,7次等等才能得到6启动游戏;第二种,前4次都没有扔出6。

而这两种解释也带来了三种求算概率的方法。


第一种:正面累加
\begin{flalign*}
P(X=5)+P(X=6)+\ldots &= \frac{5}{6}^4\times \frac{1}{6} + \frac{5}{6}^5\times \frac{1}{6}+\ldots\\
					 &=\frac{\frac{5}{6}^4\times \frac{1}{6}}{1-\frac{5}{6}}\\
					 &=\frac{5}{6}^4
\end{flalign*}
这里需要注意,其实应该是一个等比数列的sum to infinity的公式。

第二种:反面推导
用1减去小于等于4次的概率也是可以的。而且这种方法的好处是有限项。
\begin{flalign*}
1&-P(X=1)-P(X=2)+\ldots+P(X=4)\\
 &=1-\left(\frac{1}{6}+\frac{5}{6}\times\frac{1}{6}+\frac{5}{6}^2\times\frac{1}{6}+ \frac{5}{6}^3\times\frac{1}{6}\right)\\
					 &=\frac{5}{6}^4
\end{flalign*}

第三种:最巧妙的解法
在刚才的解释当中,只要前四次都是失败的,那么必定需要更多的次数。因此此题完全等同于求算连续四次失败的概率,因此求算起来最为简单
\[
	P(X>4)=(\frac{5}{6})^4
\]

这三种方案当中,无论是哪一种方法都可以帮助解决此类的问题,没有高下之分。
\end{ExampleBox}

