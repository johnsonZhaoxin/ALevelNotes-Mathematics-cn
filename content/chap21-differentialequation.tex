\chapter{微分方程}
微分方程是工程学的通行证,线性代数是计算机的奠基石

\section*{学习目标}
\begin{todolist}
	\item 根据文字陈述列出表示速率相关的微分方程
	\item 通过积分求算可分离变量的一阶微分方程的\gls{generalsolu}
	\item 根据\gls{initialcondition}求算微分方程的\gls{particularsolu}
	\item 利用微分方程解决生活中的具体问题
\end{todolist}
\clearpage

\section{微分方程的基本概念}
首先,看一下微分方程的定义:
\begin{definition}
表示一个未知的函数$y$,该函数的导数$\frac{\d y}{\d x}$,以及自变量$x$之间关系的方程
\end{definition}
也就是表达式中函数有导数$\frac{\d y}{\d x}$,$y$或者$x$的方程。这样的方程的\textbf{解}并不是一个特定值,而是$y$关于$x$的函数关系。

\subsection*{微分方程的阶}
在一个微分方程中,最高阶数的导数称之为该微分方程的阶数。比如
\[
	\frac{\d y}{\d x}=y \quad \frac{\d^2 y}{\d x^2} +\frac{\d y}{\d x}=y
\]
分别就是一阶微分方程和二阶微分方程

\subsection*{微分方程的通解}
\[
	\frac{\d y}{\d x}=y
\]
就是一个较为简单的微分方程,如果用文字性的语言就是:一个函数的导数和这个函数自身相等。那么$y=e^x$就从脑海当中浮现出来了。这样的函数关系显然满足$\frac{\d y}{\d x}=e^x=y$。因此它就是这个微分方程的解。问题解决了,但是没有完全解决,因为显然$y=5e^x$,$y=-3e^x$,$y=\sqrt{2} e^x$,甚至$y=0$也是满足要求的解。但是仔细观察,这些函数都可以$y=C\cdot e^x$这样的函数表达,其中$C$为一个待定的常数。因此这样的形式叫做微分方程的通解。

\subsection*{初始条件和特解}

\[
	\frac{\d y}{\d x}=y \quad y(0)=3
\]
$y(0)=3$给定了代求函数$y$上的点,这样的信息被我们称之为初始条件。仅需要将其带入到通解$y=C\cdot e^x$当中,就可以确定该待定系数$C=3$。同时满足微分方程的解就只剩下$y=3\cdot e^x$这样唯一的一个解,也被称之为该方程的特解。
\clearpage

\section{求解过程}
前面讲了这些基本概念,但是微分方程的求解还是没有涉及。现在开始处理\textbf{可分离的、一阶}微分方程。在A Level p3的考试当中,微分方程并不是特别复杂的。在这里的可分离的意思是$x$和$y$可以分开。
\begin{ExampleBox}
求解$\frac{\d y}{\d x}=y$
\tcblower
\begin{align*}
\frac{1}{y}\d y &= 1\d x\\
\int \frac{1}{y}\d y &= \int 1\d x\\
\ln |y| &= x+C_0\\
y &=e^{x+C_0}\\
&=e^x\cdot e^{C_0} \\
&=C\cdot e^x
\end{align*}
\end{ExampleBox}

\begin{TaskBox}
在上面的过程当中,稍微有一些跳步,比如$y$的正负的讨论。可以自己尝试一下,如果$y$是负数的情况,最终的通解形式应该是如何的。
\end{TaskBox}

因此。整理一下任何\textbf{可分离的,一阶}微分方程都是可以通过各种运算将导数分解为两个微分:
\[
	g(y) \d y =f(x) \d x
\]
的形式。其中,$g(y)$是关于$y$的函数,$f(x)$是关于$x$的函数/。下一步就是两边同时做积分,得到:
\[
	\int g(y) \d y =\int f(x) \d x
\]
因此最后结果为:
\[
	G(y) =F(x)+C
\]
到这一步基本上就算解决完毕了。只需要将$y$整理分离就行,这样就可以得到通解了。

\begin{TaskBox}
The variables $x$ and $\theta$ satisfy the differential equation
\[
	x\cos^2\theta \frac{\d x}{\d \theta} =2\tan\theta +1,
\]
for $0\leqslant \theta \leqslant \frac{1}{2}\pi$ and $x > 0$. It is given that $x=1$ when $\theta =\frac{1}{4}\pi$. Find the particular solution to the differential equation.
\makebox{}\hfill Adapted from 2018 summer qp32 Q6
\end{TaskBox}
\clearpage


\section{微分方程在实际场景的运用}
就像之前所说,微分方程是工程学的通行证,原因就是在实际情景中很多关系都可以通过微分关系进行描述,比较经典的有降温曲线,人口的\href{https://www.maa.org/press/periodicals/loci/joma/logistic-growth-model-background-logistic-modeling}{逻辑斯谛增长},化学元素的衰变等等都是通过最基本的微分方程发现的规律。

\subsection*{速率作为对时间的导数}
上面提及的场景中当中,第一步就是针对速率进行导数的确定,这里的速率也就是我们所说的``The
 rate of change of some variables'',因此就是某一个变量对时间的\textbf{导数}。比如人口的变化率可以计作$\frac{\d P}{\d t}$,温度的变化率可以计作$\frac{\d T}{\d t}$。这里的变量都是具有特定意义的,可以认为是与时间有关的函数。

\subsection*{应用类问题的基本思路}
这一类的问题是微分方程的常见考察方式,会在文字当中描述微分关系,然后再给定初始条件,进而求算最终的结果。整个流程可以分为以下几步:
\begin{SummBox}
\begin{itemize}
	\item 根据文字表述,列出变量的微分方程,如果文字当中有出现\textbf{propportional}这样的字眼,需要增加比例系数$k$,如果是下降的速率的话,微分方程当中等于号的右边必须加负号,来确保导数小于$0$
	\item 通过分离变量,再积分的方法求解微分方程的通解
	\item 带入给定的初始条件或者其他信息,确定该微分方程的特解,借此确定函数关系
	\item 一旦变量的与时间的关系确定之后,就可以求算其他的信息
\end{itemize}
\end{SummBox}

一条完整的例题进行讲解
\begin{ExampleBox}
In a certain chemical reaction, a compound A is formed from a compound B. The masses of A and B at time t after the start of the reaction are $x$ and $y$ respectively and the sum of the masses is equal to $50$ throughout the reaction. At any time the rate of increase of the mass of A is proportional to the mass of B at that time.

Explain why $\frac{\d y}{\d t}=k(50-x)$ , where $k$ is a constant.

It is given that $x = 0$ when $t = 0$, and $x = 25$ when $t = 10$.

Solve the differential equation and express $x$ in terms of $t$.
\makebox{}\hfill Adapted from 2017 summer qp33 Q8
\tcblower

Because as stated in the text ``the increase of mass is proportional to the mass of B at that time'', and the sum of masses of A and B is $50$. thus $\frac{\d x}{\d t} = k y$ and $y=50-x$

\begin{align*}
\frac{\d x}{\d t} &= k(50-x)\\
\frac{1}{50-x} \d x &= k\d t\\
\int \frac{1}{50-x} \d x &= \int k\d t\\
\ln |50-x| &= kt + C\\
50-x &=C\cdot e^{kt}\\
\end{align*}
plugging in $x=0$ when $t=0$ and $x=25$ when $t=10$:
\begin{cases}
50-0 = C\cdot e^{k\cdot 0}\\
50-25 = C\cdot e^{k\cdot 10}
\end{cases}
thus $C=50$ , $k=-0.0693$.

the amount of A is $x=50-50\cdot e^{-0.0693 t}$ 
\end{ExampleBox}

