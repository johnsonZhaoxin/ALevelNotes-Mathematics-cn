%% glossaries

%% Chapter 1

\newglossaryentry{discriminant}{
  name = {判别式},
  description = {\emph{discriminant} $\Delta = b^2-4ac$ 用来确定二次函数的根的个数的表达式,当$\Delta > 0$时,二次方程有两个不一样的实根;当$\Delta = 0$时,二次方程有两个相等的实根,或者称之为一个根;$\Delta < 0$时,二次方程没有实数解,但是有两个共轭的复数解}
}


\newglossaryentry{root}{
  name = {根},
  description = {\emph{root}变量的特定值,能够使方程成立,或者满足方程。也被称之为方程的解} 
}

\newglossaryentry{quar form}{
  name={求根公式},
  description={\emph{quadratic formula}形如$ax^2+bx+c=0$的二次方程的根的通用表达式}
}


\newglossaryentry{sym axis}{
  name={对称轴},
  description={\emph{axis of symmetry}二次函数的中心轴}
}

\newglossaryentry{coeff}{
  name={系数},
  description={\emph{coefficient} 多项式表达式当中次方项前所乘的数字,比如 $4x^5$ 当中的$4$}
}

\newglossaryentry{lequ}{
  name={线性方程},
  description={\emph{Linear Equation} 只含有一个未知数,并且次数仅为$1$的方程,比如 $2x+4=-2$}
}

\newglossaryentry{QED}{
		  name={QED},
		  description={\emph{QED} Quod Erat Demonstrandum 证明完毕,仅此作答 }
}


\newglossaryentry{LHS}{
		  name={LHS},
		  description={\emph{LHS} Left Hand Side,表达式的左边}
}

\newglossaryentry{RHS}{
		  name={RHS},
		  description={\emph{RHS} Right Hand Side,表达式的右边}
}	

\newglossaryentry{conjugate}{
  name={共轭},
  description={\emph{Conjugate} 实部相同,虚部相反的复数,会是同一个二次方程的复数解}
}

\newglossaryentry{pwx}{
  name={抛物线},
  description={\emph{parabola} 圆锥曲线的一种,也是二次函数的图像,具有高度的对称性}
}

\newglossaryentry{vertex}{
  name={顶点},
  description={\emph{vertex}二次函数图像上仅有唯一的函数值的地方,也是与对称轴的交点}
}

\newglossaryentry{intercept}{
  name={截距},
  description={\emph{intercept} 截距是任意函数图像与$x$轴或者$y$轴相交点的坐标}
}

\newglossaryentry{fcz}{
  name={方程式组},
  description={\emph{Simultaneous Equations}含有多个未知数、变量的的多个等式}
}

%Chapter2
\newglossaryentry{set}{
  name={集合},
  description={\emph{set}满足要求的个体全部组合}
}

\newglossaryentry{elem}{
  name={元素},
  description={\emph{element} 集合中每一个单独的个体}
}

\newglossaryentry{mapping}{
  name={映射},
  description={\emph{映射}两个集合之间的元素的对应关系}
}

\newglossaryentry{zbl}{
    name={自变量},
    description={\emph{independent variable}可以任意改变的变量}
} 

\newglossaryentry{ybl}{
    name={因变量},
    description={\emph{dependent variable}自变量中对应的值}
}  

\newglossaryentry{domain}{
  name={定义域},
  description={\emph{domain}函数自变量的取值范围}
}

\newglossaryentry{range}{
  name={值域},
  description={\emph{range}函数因变量的取值范围}
}


\newglossaryentry{inverse}{
  name={反函数},
  description={\emph{inverse function}将某一函数的对应关系进行对调之后形成的函数}
}

\newglossaryentry{comp}{
  name={复合函数},
  description={\emph{composite function}自变量通过一个函数运算之后,再将该值作为新函数的自变量产生的最终结果的函数。}
}


\newglossaryentry{py}{
  name={平移},
  description={\emph{translation}图形没有改变形状,发生整体的移动}
}




%Chapter4 坐标几何
\newglossaryentry{xjs}{
  name={斜截式},
  description={\emph{slope intercept form} 提供直线的斜率和截距的形式}
}
\newglossaryentry{dxs}{
  name={点斜式},
  description={\emph{slope intercept form}提供直线经过的点以及斜率的表达形式}
}

\newglossaryentry{ybs}{
  name={一般式},
  description={\emph{standard form} 能够用于描述平面上任何直线的形式,包括没有斜率的直线}
}

\newglossaryentry{bzfc}{
    name={圆的标准方程},
    description={\emph{standard equation of a circle} 形如$(x-a)^2+(y-b)^2=r^2$的方程。利用圆的定义和距离公式推导}
}

\newglossaryentry{slope}{
    name={斜率},
    description={\emph{gradient}线性函数的系数}
}

\newglossaryentry{jj}{
       name={截距},
       description={\emph{intercept}线性函数的常数项}
}

\newglossaryentry{radius}{
  name={半径},
  description={\emph{radius}圆的半径}
}


\newglossaryentry{radians}{
  name={弧度},
  description={\emph{radian} 用弧长与半径之比表示圆心角大小的计量方式}
}

\newglossaryentry{pangle}{
  name={平角},
  description={\emph{straight angle}两边都在一条直线上的角,其大小被定义为180\si{\degree}}
}


%chapter5 Trigonometry
\newglossaryentry{inverse trig}{
  name={反三角函数},
  description={\emph{inverse trigonometry}根据某个角度三角比,求算其角度的函数,是三角函数的反函数}
}


\newglossaryentry{pascaltri}{
  name={杨辉三角},
  description={\emph{Pascal's Triangle} 每个数等于它上方两数之和从而构成的三角形,可以用来确定系数。在西方由法国数学家\href{https://en.wikipedia.org/wiki/Blaise_Pascal}{Blaise Pascal}发现并进行研究运用。因此命名为Pascal's Triangle,实际上印度,中国,德国,意大利都有类似的研究。}
}


\newglossaryentry{combi}{
    name={组合数},
    description={\emph{combination}组合数,计作$nCr$的形式或者${n\choose r}$用来确定从$n$个物体当中,无顺序地选择$r$个物体的挑选方案总数的计算方法}
}

\newglossaryentry{factorial}{
  name={阶乘},
  description={\emph{factorial}小于及等于某数的正整数的积,计作$n!$}
}


\newglossaryentry{dcsl}{
  name={等差数列},
  description={\emph{arithmetic sequence} 后一项总是比前一项相差恒定值的数列}
}

\newglossaryentry{commondiff}{
  name={公差},
  description={\emph{common difference} 等差数列当中后一项与前一项的固定差值,可正可负,也可以为0}
}

\newglossaryentry{dbsl}{
    name={等比数列},
    description={\emph{geometric sequence}后一项是前一项固定倍数的数列}
}

\newglossaryentry{commonratio}{
  name={公比},
  description={\emph{common ratio}等比数列中,后一项与前一项之比,为固定值}
}

\newglossaryentry{generalterm}{
  name={通项公式},
  description={\emph{general term}用来概括任何数列的第$n$项与其对应的数据值的表达式。}
}


\newglossaryentry{converge}{
  name={收敛},
  description={\emph{converge}无穷级数逼近的极限}
}

\newglossaryentry{binomial}{
  name={二项式},
  description={\emph{binomial}只含有两项表达式,可以加$n$次方}
}

\newglossaryentry{stationary}{
  name={驻点},
  description={\emph{stationary point}函数图像上切线为水平线的点}
}

\newglossaryentry{chainrule}{
  name={链式法则},
  description={\emph{chain rule}求导的法则之一,利用微分的传递性}
}

\newglossaryentry{normal}{
  name={法线},
  description={\emph{normal line}与切线互相垂直的直线}
}

\newglossaryentry{tangent}{
  name={切线},
  description={\emph{tangent line}刚好碰触到函数图像的一条之间,碰触点称之为切点}
}

\newglossaryentry{derivative}{
  name={导数},
  description={\emph{derivative}函数在某一点处的切线的斜率}
}

\newglossaryentry{mono}{
  name={单调性},
  description={\emph{monotonicity}函数的增减性}
}


\newglossaryentry{substitution}{
  name={换元法},
  description={\emph{integration by substitution}将被积部分通过等式关系更换为其他的变量进行积分的方法,有第一类和第二类换元法}
}

\newglossaryentry{dinte}{
  name={定积分},
  description={\emph{definite integral}有明确上下限的积分,结果为固定数值}
}

\newglossaryentry{iinte}{
  name={不定积分},
  description={\emph{indefinite integral}积分号没有明确上下限的积分运算,结果为一系列同族的函数表达式}
}

\newglossaryentry{improper inte}{
  name={广义积分},
  description={\emph{improper integral}积分的上下限范围中包含一个或多个瑕点的定积分}
}


%chap10
\newglossaryentry{limited equil}{
  name={受限平衡},
  description={\emph{limited equilibrium}由于静摩擦力存在导致的平衡,但是由于静摩擦力有上限,因此该平衡有可能被打破}
}

\newglossaryentry{fricoeff}{
  name={摩擦系数},
  description={\emph{coefficient of friction}是摩擦力与支持的力的比值,与接触面的粗糙程度有关}
}

\newglossaryentry{force}{
  name={力},
  description={\emph{force}是物理学的鼻祖概念}
}

\newglossaryentry{component}{
  name={分量},
  description={\emph{component}组合成一个分量的所有部分}
}

\newglossaryentry{vector}{
  name={矢量},
  description={\emph{vector}具有大小以及方向的量}
}

\newglossaryentry{contact}{
  name={接触力},
  description={\emph{contact force}物体之间互相接触带来的力}
}

\newglossaryentry{weight}{
  name={重力},
  description={\emph{weight}地球对地表物体的吸引}
}

\newglossaryentry{acdueg}{
  name={重力加速度},
  description={\emph{acceleration due to gravity}}
}

\newglossaryentry{normalf}{
  name={支持力},
  description={\emph{Normal Contact Force}和接触面互相垂直的作用力}
}


%chap10

\newglossaryentry{rframe}{
  name={参考系},
  description={\emph{Reference Frame}由于运动的相对性问题,在描述物体运动之前必须明确这个运动物体是相对于哪个物体或者体系来观察其运动状态,这个被选作参照的物体就是参考系}
}

\newglossaryentry{kinematics}{
  name={运动学},
  description={\emph{Kinematics}研究物体或者系统运动状态的物理学分支,是物理学当中最简单的最入门的学科}
}

\newglossaryentry{positionvec}{
  name={位矢},
  description={\emph{Position Vector}是一个瞬时状态物理量,是指某一个时刻从参考系的原点指向运动的物体的所处位置的矢量}
}

\newglossaryentry{distance}{
  name={距离},
  description={\emph{Distance}指运动轨迹的总长度,如果运动方向一直在改变的话,需要考虑几何方法计算曲线长度}
}

\newglossaryentry{displacement}{
  name={位移},
  description={\emph{Displacement}指一段时间从运动起点指向运动终点的矢量,和位矢有关,但是是一个过程量}
}

\newglossaryentry{linearmotion}{
  name={线性运动},
  description={\emph{Linear Motion}指运动轨迹为直线的运动,其速度方向和加速度方向必须在同一条直线上}
}

\newglossaryentry{speed}{
  name={速率},
  description={\emph{Speed}指运动距离变化快慢的物理量,瞬时速率的就是瞬时速度的大小}
}

\newglossaryentry{velocity}{
  name={速度},
  description={\emph{Velocity}指位移变化快慢的物理量,由于位移是矢量,因此速度也是矢量,在描述物体运动速度的时候必须要指明速度的方向}
}

\newglossaryentry{insv}{
  name={瞬时速度},
  description={\emph{Instantaneous Velocity}就是物体在某个时刻或者经过某个位置的速度}
}


\newglossaryentry{acc}{
  name={加速度},
  description={\emph{Acceleration}是描述物体速度变化快慢的物理量}
}

\newglossaryentry{uni-mo}{
  name={匀速直线运动},
  description={\emph{Uniform Motion}}
}

\newglossaryentry{uni-acc}{
  name={匀加速直线运动},
  description={\emph{Uniformly Accelerated Motion}}
}


%chap11 动量
\newglossaryentry{momentum}{
  name={动量},
  description={\emph{momentum}物体的速度与质量的乘积,是一个矢量}
}

\newglossaryentry{impulse}{
  name={冲量},
  description={\emph{impulse} 力与时间的积分,是一个矢量}
}

\newglossaryentry{cons momentum}{
  name={动量守恒},
  description={\emph{conservation of momentum}在某个过程中,如果没有外力,系统的总动量维持不变,在任何时刻都保持一致}
}

\newglossaryentry{accdueg}{
  name={重力加速度},
  description={\emph{acceleration due to gravity}在仅收到重力作用情况下物体所具有的加速度}
}

\newglossaryentry{freefall}{
  name={自由落体},
  description={\emph{free fall}仅受到重力作用,因为以重力加速度做匀加速向下的运动}
}

%chap13功能与能量
\newglossaryentry{work}{
  name={做功},
  description={\emph{work}力和力所通过的位移的乘积}
}

\newglossaryentry{gpe}{
  name={重力势能},
  description={\emph{gravitational potential energy}与物质所处的高度有关的能量}
}

\newglossaryentry{reflevl}{
  name={参考面},
  description={\emph{reference level}计算重力势能的参考面,默认出在该位置上物体的重力势能为$0$}
}

\newglossaryentry{ke}{
  name={动能},
  description={\emph{kinetic energy}与物体的运动有关的能量}
}

\newglossaryentry{consforce}{
  name={保守力},
  description={\emph{conservative force}做功的大小与路径无关的作用力,比如重力,弹簧的弹力等}
}

\newglossaryentry{series}{
  name={级数},
  description={\emph{series}数列当中每一项之和}
}

\newglossaryentry{power}{
  name={功率},
  description={\emph{power}做功的速率}
}


%chap 14 further algebra
\newglossaryentry{abs}{
  name={绝对值},
  description={\emph{absolute value}表示某个数字或者某个代数式在数轴上到原点的距离}
}

\newglossaryentry{remaindertheo}{
  name={余数理论},
  description={\emph{Remainder Theorem}快速求算多项式被低次多项式所除的余数的方法}
}

\newglossaryentry{polynomial}{
    name={多项式},
    description={\emph{polynomial}每一项都是系数乘以变量的次方相加得到的表达式}
  }  

\newglossaryentry{factortheo}{
  name={因式定理},
  description={\emph{Factor Theorem}判断线性表达式是否为多项式的因式的定理}
}


\newglossaryentry{partialfrac}{
  name={有理分式分解},
  description={\emph{Partial Fraction}可以将任意的有理式分解成为低次的有理式之和、差的形式}
}


%15 log exp
\newglossaryentry{base}{
  name={底},
  description={\emph{base}幂次运算的基础对象,也是取对数的运算基底}
}

\newglossaryentry{expo}{
  name={幂次},
  description={\emph{exponents}指数部分}
}

\newglossaryentry{log}{
  name={对数},
  description={\emph{logarithm}求算幂次的运算}
}

\newglossaryentry{changebase}{
  name={换底公式},
  description={\emph{change of base}将任意对数表示为两个相同底的对数之商的形式}
}

\newglossaryentry{Euler's number}{
  name={自然对数},
  description={\emph{Euler's number} 2.718281828459045...}
}

\newglossaryentry{asymptote}{
  name={渐近线},
  description={\emph{asymptote}函数图像越来越靠近的直线,也可以认为是这个图像的边界}
}

\newglossaryentry{sec}{
  name={正割},
  description={\emph{secant}直角三角形中的斜边与邻边之比}
}

\newglossaryentry{csc}{
  name={余割},
  description={\emph{cosecant}直角三角形中的斜边与对边之比}
}

\newglossaryentry{cot}{
  name={余切},
  description={\emph{cotangent}直角三角形中的临边与对边之比}
}


%chap 16 further differentiation
\newglossaryentry{product}{
  name={乘法法则},
  description={\emph{product rule}用来求算两个函数相乘的结果的导数}
}


\newglossaryentry{quotient}{
  name={除法法则},
  description={\emph{quotient rule} 用来计算两个函数相除之后的结果的导数}
}

\newglossaryentry{parafunc}{
  name={参数函数},
  description={\emph{parametric function}通过参数$t$定义自变量和因变量的函数}
}

\newglossaryentry{impfunc}{
  name={隐函数},
  description={\emph{implicit function}自变量和因变量难以分离,通过等式定义的函数}
}


%chap 17 further integration
\newglossaryentry{u-sub}{
  name={换元法},
  description={\emph{u-substitution}通过更换微元,将复杂的积分变成容易积分的方法}
}


\newglossaryentry{bypart}{
  name={分部积分法},
  description={\emph{integration by part}乘积法则}
}

\newglossaryentry{numsolution}{
  name={数值解},
  description={\emph{numerical solution}通过有限元法分析,逼近法,二分法,牛顿迭代法等暴力破解方式所求出的方程的解}
}

\newglossaryentry{jiezhi}{
  name={介值定理},
  description={\emph{Intermediate Value Theorem}连续函数在封闭区间的值域必定包含了介于两个端点值之间的所有实数}
}

\newglossaryentry{bolzano}{
  name={博尔扎诺定理},
  description={\emph{Bolzano's Theorem}连续函数在封闭区间上,如果端点值一正一负,则在该区间内至少包含有一个根}
}

\newglossaryentry{dichotomy}{
  name={二分法},
  description={\emph{dichotomy}通过将区间等分不断缩小根所在的范围,进而逼近该方程的根的方法}
}


\newglossaryentry{iteration}{
  name={迭代},
  description={\emph{iteration}从已有的推算下一次,并且将这种过程重复进行,直到满足要求}
}


%chap 20 Vector
\newglossaryentry{dotproduct}{
  name={点乘},
  description={\emph{Scalar Product}矢量点乘的结果}
}

\newglossaryentry{crossproduct}{
  name={叉乘},
  description={\emph{Cross Product}矢量叉乘的结果}
}

\newglossaryentry{columnvec}{
  name={列向量},
  description={\emph{Column Vector}将矢量的各单位分量写成竖排的形式}
}

\newglossaryentry{modulus}{
  name={模},
  description={\emph{Modulus}矢量的大小}
}

\newglossaryentry{unitvector}{
  name={单位矢量},
  description={\emph{Unit Vector}与该矢量方向完全一致,且模为$1$的矢量}
}

\newglossaryentry{scalar}{
  name={标量},
  description={\emph{Scalar}只有大小,没有方向的量}
}

\newglossaryentry{pos vec}{
  name={位置矢量},
  description={\emph{position vector}相对于指定一点的,可用于描述空间当中的其他点的位置的矢量}
}

\newglossaryentry{skew}{
  name={异面},
  description={\emph{skew}两条直线没有交点,但不处于同一个平面上}
}

%chapter 21 differential equations


%22章 Complex Number
\newglossaryentry{imag}{
  name={虚数},
  description={\emph{Imaginary Unit} 一个特定的数字$i$,满足$i^2=-1$}
}

\newglossaryentry{complex}{
  name={复数},
  description={\emph{Complex Number}由实数和虚数复合而成的数字}
}
\newglossaryentry{arg}{
  name={幅角},
  description={\emph{Argument}复平面上表示复数的矢量与实数数轴的正方向产生的夹角}
}

\newglossaryentry{conj}{
  name={共轭(复数)},
  description={\emph{Conjugate Complex}实部相同,虚部相反的一对复数}
}

\newglossaryentry{arg diagram}{
  name={阿干特图},
  description={\emph{Argand Diagram}用复平面当中的点表示复数的图形}
}

\newglossaryentry{loci}{
  name={轨迹图},
  description={\emph{locus/loci}满足要求的复数形成的轨迹}
}

\newglossaryentry{generalsolu}{
  name={通解},
  description={\emph{General Form of Solution}微分方程的解当中含有与其对应方程阶数相等数目的待定系数或者常数,这样的形式能够用于描述一组解,称之为通解}
}

\newglossaryentry{initialcondition}{
  name={初始条件},
  description={\emph{Initial Condition} 微分方程当中给定的函数的值}
}

\newglossaryentry{particularsolu}{
  name={特解},
  description={\emph{Particular Solution}满足给定的初始条件的微分方程的特定的唯一解}
}


%Chap 23 数据呈现
\newglossaryentry{stem-leaf}{
  name={茎叶图},
  description={\emph{stem-and-leaf diagram}将数据的共有部分作为茎干,个位数字作为枝叶,按照顺序排布的图像}
}

\newglossaryentry{boxwhisker}{
  name={盒须图},
  description={\emph{box-and-whisker plot}盒须图、盒式图或箱线图}
}

\newglossaryentry{histogram}{
  name={直方图},
  description={\emph{histogram}}
}

\newglossaryentry{cumulative}{
  name={累积频数分布图},
  description={\emph{英文全称}展现小于或者等于特定值的频数的曲线图像}
}

\newglossaryentry{lower quartile}{
  name={下四分位数},
  description={\emph{lower quartile}对数据进行从小到大的排序,位于序列1/4处的数据}
}

\newglossaryentry{upper quartile}{
  name={上四分位数},
  description={\emph{upper quartile}对数据进行从小到大的排序,位于序列3/4处的数据}
}

\newglossaryentry{IQR}{
  name={四分位距},
  description={\emph{InterQuartile Range}上四分位数与下四分位数之差,代表不大不小的,占据全体统计值中间一半的数据个数}
}

\newglossaryentry{outlier}{
  name={离群值},
  description={\emph{outlier}与统计结果中其他数据相差过大的数据值,也可以称之为异常值}
}

\newglossaryentry{onevariable}{
  name={单变量统计学},
  description={\emph{One Variable Statistics}只记录一个变量的统计学}
}

\newglossaryentry{frequencytable}{
  name={频率分布表},
  description={\emph{frequency table}将统计值和该统计值出现的频数(率)作为一行的表格}
}

\newglossaryentry{classwidth}{
  name={组距},
  description={\emph{class width}频率分布表或者直方图当中每个区间的上限减去下限之差}
}

\newglossaryentry{freqdensity}{
  name={频率密度},
  description={\emph{Frequency Density}频率密度,等同于频率除以组距}
}

\newglossaryentry{mean}{
  name={平均值},
  description={\emph{mean/arithmatic average}}
}

\newglossaryentry{median}{
	name={中位数},
	description={\emph{median}}
}

\newglossaryentry{mode}{
	name={众数},
	description={\emph{mode}}
}

\newglossaryentry{skewness}{
  name={斜度},
  description={\emph{skewness}指数据分布的不对称性}
}

\newglossaryentry{kurtosis}{
	name={峰度},
	description={\emph{kurtosis}指数据分布的集中性程度}
}

\newglossaryentry{var}{
  name={方差},
  description={\emph{Variance} Var 或者 $\sigma^2$,一组数据的方差}
}

\newglossaryentry{sd}{
  name={标准差},
  description={\emph{standard deviation}s.d. 或者 $\sigma$, 一组数据的标准差}
}


% Chap 24 nPr nCr
\newglossaryentry{permu}{
  name={排列},
  description={\emph{permutation}计算有序元素排布方式的数学方法}
}


% Chap 25 概率
\newglossaryentry{event}{
  name={随机事件},
  description={\emph{event}概率论当中的基本概念,是指一个实验出现的结果的合集。}
}

\newglossaryentry{independ}{
  name={独立事件},
  description={\emph{independent events}没有关联,互不影响的事件}
}

\newglossaryentry{mutualexclusive}{
  name={互斥事件},
  description={\emph{mutually exclusive events} 互相排斥的事件,一个发生,另一个就绝对不会发生}
}

\newglossaryentry{condi}{
  name={条件概率},
  description={\emph{conditional probability} 在一件事$B$已经发生的前提下,$A$事件再发生的概率,也被称之为后验概率}
}

\newglossaryentry{joint}{
  name={联合事件},
  description={\emph{joint probability} 多个事件同时发生的事件}
}

\newglossaryentry{opposite}{
  name={对立事件},
  description={\emph{opposite events} 两个事件当中必定只能发生其中的一个}
}


\newglossaryentry{bayes}{
  name={贝叶斯定理},
  description={\emph{Bayes Theorem} 从后验概率推导先验概率的公式}
}

\newglossaryentry{classicprob}{
  name={古典概型},
  description={\emph{classic probability}样本空间为有限元素的集合,通过满足事件的元素个数比上样本空间内全部元素的个数求算的概率}
}

%chap26 离散随机变量
\newglossaryentry{randomvari}{
  name={随机变量},
  description={\emph{Random Variable or Stochastic Variable)}用于表示随机试验结果的变量,一般用大写字母表示}
}

\newglossaryentry{expectation}{
  name={期望},
  description={\emph{Expectation}期望}
}

\newglossaryentry{variance}{
  name={方差},
  description={\emph{Variance}方差}
}

\newglossaryentry{probdist}{
  name={概率分布表},
  description={\emph{Probability Distribution Table}The probability distribution of a discrete random variable is a listing of the possible values of the variable and the corresponding probabilities.}
}

\newglossaryentry{0-1distr}{
  name={0-1分布},
  description={\emph{0-1 Distribution} 随机试验结果只有两种的随机变量分布}
}

% Chap 26 Binomial
\newglossaryentry{bidistr}{
  name={二项分布},
  description={\emph{Binomial Distribution}}
}

\newglossaryentry{prob density}{
  name={概率密度},
  description={\emph{Probability Distribution}概率除以随机变量的跨度}
}
\newglossaryentry{geodistr}{
    name={几何分布},
    description={\emph{Geometric Distribution}}
}

\newglossaryentry{bernoullidistr}{
  name={伯努利分布},
  description={\emph{Bernoulli Distribution} $0-1$分布的另外一个名字}
}

\newglossaryentry{trial}{
  name={试验},
  description={\emph{Trial}一个随机实验当中的一次基本试验}
}

% Chap 28 normal distribution
\newglossaryentry{bellcurve}{
  name={钟形曲线},
  description={\emph{Bell Like Curve}正态分布的图像曲线}
}

\newglossaryentry{snd}{
  name={标准正态分布},
  description={\emph{Standard Normal Distribution}标准正态分布是期望为0,方差为1的正态分布,从该分布推导了标准正态分布表}
}

\newglossaryentry{standardize}{
  name={标准化},
  description={\emph{Standardize/Standardization}将任意正态分布分转变为标准的正态分布的方法}
}