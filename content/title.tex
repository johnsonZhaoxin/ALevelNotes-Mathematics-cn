% file: title.tex title界面
\begin{titlepage} % Suppresses headers and footers on the title page
\begin{tikzpicture}[remember picture, overlay]
	\draw[draw=none,fill=r1] (current page.north west) rectangle
		($(current page.north west)+(\paperwidth,-4\paperheight/10)$);

	\draw[draw=none,fill=g1] ($(current page.north west)+(0,-2\paperheight/5)$) rectangle
		($(current page.north west)+(\paperwidth,-7\paperheight/10)$);

	\draw[draw=none,fill=b1] ($(current page.south west)+(0,\paperheight/10)$) rectangle
		($(current page.south east)+(0,3\paperheight/10)$);

	\draw[draw=none,fill=p1] (current page.south west) rectangle
		($(current page.south east)+(0,\paperheight/10)$);

	\draw[anchor=west] ($(current page.south west)+(0.125\paperwidth,5\paperheight/12)$) node[yshift=3ex]
		{\sf\Huge\textcolor{white}{A-Level数学笔记}};
		
	\draw[anchor=west] ($(current page.south west)+(0.125\paperwidth,5\paperheight/12)$) node[yshift=-6ex]
		{\kaishu\LARGE\textcolor{white}{赵三金瞎写的}};
	\draw[anchor=west] ($(current page.south west)+(0.125\paperwidth,\paperheight/20)$) node
        {\tikz \fill[color=white,draw=none,x=1mm,y=1mm] (0,0) -- (0,5) -- (3,5) -- (3,1) -- (4,1) -- (4,6) -- (0,6) -- (0,7) -- (4,7) -- (4,8) -- (0,8) -- (0,9) -- (9,9) -- (9,8) -- (5,8) -- (5,7) -- (9,7) -- (9,6) -- (5,6) -- (5,1) -- (6,1) -- (6,5) -- (9,5) -- (9,0) -- (8,0) -- (8,4) -- (7,4) -- (7,0) -- (2,0) -- (2,4) -- (1,4) -- (1,0) -- cycle;};
\end{tikzpicture}

\end{titlepage}
